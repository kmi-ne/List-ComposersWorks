\documentclass[autodetect-engine, ja=standard, base=9pt, b5j, label-section=none]{bxjsarticle}

\usepackage{../settings}

\begin{document}

\mytitle{Enescu, George}{1881}[1955]

\myintro{ジョルジェ・エネスク(George Enescu)}{Enescu,_George}

\myytlist{https://www.youtube.com/playlist?list=PLRUhD6IuubZ_H_dLLmoS0DANyPLyzzviI}{https://www.youtube.com/playlist?list=PLRUhD6IuubZ8NV8UQjnW6iIiQXAKP4Miz}

\section*{作品一覧}
\addcontentsline{toc}{section}{作品一覧}

\infotitle{2024-07-09}
{オペラ}{\textit{Opera}}
{1886}
\info{
  yeardate_ref ={\cite{Mal_90}},
  instrumentation ={ヴァイオリン~1,ピアノ~1},
  instrumentation_ref ={\cite{Mal_90}},
  supplement ={エネスクがラジオインタビューで演奏.全24小節.\supcite[31]{Mal_90}},
}

\infotitle{2024-07-09}
[Fiori_de_Garoafa]{カーネーション}{\textit{Fiori de Garoafa}}
{1888?}
\info{
  yeardate_ref ={\cite{Mal_90, PMG_16}},
  instrumentation ={ピアノ~1},
  instrumentation_ref ={\cite{Mal_90}},
  recording_url ={\url{https://www.youtube.com/watch?v=3KOKK5qY7MY}},
  recording_info ={\cite[tracks~1--3]{PMG_16}.Luiza Borac(ピアノ).},
  supplement ={\cite[31]{Mal_90}によれば,エネスクはラジオインタビューで1887年作曲の`Waltz'を演奏した.それ以外にエネスクがワルツを作曲したとの記載はない.一方,\cite{PMG_16}は,エネスクが1950年のラジオインタビューで1888年作曲の本作を演奏したと記している.年は異なるものの,両者は同一作品を指しているものと推定した.\nameref{Waltz_in_D_major}も参照.},
}

\infotitle{2024-07-15}
[Waltz_in_D_major]{ワルツ\ ニ長調}{Waltz in D major}
{1888}
\info{
  yeardate_ref ={\cite{PMG_16}},
  instrumentation ={ピアノ~1},
  instrumentation_ref ={\cite{PMG_16}},
  recording_url ={\url{https://www.youtube.com/watch?v=3KOKK5qY7MY}},
  recording_info ={\cite[tracks~1--2]{PMG_16}.Luiza Borac(ピアノ).},
  supplement ={\cite{PMG_16}のみが言及.\cite{Mal_90}に記載されている`Waltz'は,本作である可能性もある.\nameref{Fiori_de_Garoafa}も参照.},
}

\infotitle{2024-07-09}
{教会小品}{\textit{Pièce d'èglise}}
{1889}
\info{
  yeardate_ref ={\cite{Mal_90}},
  instrumentation ={ピアノ~1},
  instrumentation_ref ={\cite{Mal_90}},
}

\infotitle{2024-07-09}
{幻想曲}{\textit{Fantaisie}}
{1891?}
\info{
  yeardate_ref ={\cite{Mal_90}},
  supplement ={散逸.\supcite[34]{Mal_90}},
}

\infotitle{2024-07-09}
[2_overtures_for_orchestra]{管弦楽のための前奏曲\ 2曲}{2 overtures for orchestra}
{1891}
\info{
  yeardate_ref ={\cite[44]{Mal_90}},
  instrumentation ={管弦楽(詳細不明)},
  instrumentation_ref ={\cite{Mal_90}},
  supplement ={この2作か\nameref{overture_for_orchestra}のうち一部がラジオインタビューで演奏された.\supcite[44]{Mal_90}},
}

\infotitle{2024-07-20}
{伝説}{Légende}
{1891}
\info{
  yeardate_ref = {\cite{Toc_12}},
  instrumentation = {ヴァイオリン~1,ピアノ~1},
  instrumentation_ref = {\cite{Toc_12}},
  recording_url = {\url{https://www.youtube.com/watch?v=p7R78yiTN4E}},
  recording_info = {\cite[track~14]{Toc_12}.Șerban Lupu(ヴァイオリン),Ian Hobson(ピアノ).},
  supplement = {\cite{Toc_12}のみが言及.},
}

\infotitle{2024-07-09}
{ロンドと変奏}{\textit{Rondo and variations}}
{1893}
\info{
  yeardate_ref ={\cite{Mal_90}},
  instrumentation ={ピアノ~1},
  instrumentation_ref ={\cite{Mal_90}},
}

\infotitle{2024-07-09}
{ピアノ四重奏曲}{Piano quartet}
{1893}
\info{
  yeardate_ref ={\cite{Mal_90}},
  instrumentation ={ピアノ四重奏(詳細不明)},
  supplement ={断片.\supcite{Mal_90}},
}

\infotitle{2024-07-15}
[overture_for_orchestra]{管弦楽のための前奏曲}{Overture for orchestra}
{1894}
\info{
  yeardate_ref ={\cite{Mal_90}},
  instrumentation ={管弦楽(詳細不明)},
  instrumentation_ref ={\cite{Mal_90}},
  supplement ={ウィーン時代最終年の作品.本作か\nameref{2_overtures_for_orchestra}のうち一部がラジオインタビューで演奏された.\supcite[44]{Mal_90}},
}

\infotitle{2024-07-20}
{スケルツォ}{\textit{Scherzo}}
{1894}
\info{
  yeardate_ref ={\cite[About This Recording]{HNH_16_nax}},
  instrumentation ={ピアノ~1},
  instrumentation ={\cite{HNH_16_nax}},
  recording_url ={\url{https://www.youtube.com/watch?v=k61eloh4wr4}},
  recording_info ={\cite[track~1]{HNH_16}.Josu de Solaun(ピアノ).},
  supplement ={\cite{HNH_16}のみが言及.}
}

\infotitle{2024-07-09}
{バラード}{\textit{Ballade}}
{1894}
\info{
  yeardate_ref ={\cites{Mal_90}[About This Recording]{HNH_16_nax}},
  instrumentation ={ピアノ~1},
  instrumentation_ref ={\cite{Mal_90, HNH_16_nax}},
  recording_url ={\url{https://www.youtube.com/watch?v=1qXUq8k_qF0}},
  recording_info ={\cite[track~2]{HNH_16}.Josu de Solaun(ピアノ).},
}

\infotitle{2024-07-09}
{序奏,アダージョ,アレグロ}{\textit{Introduction, adagio and allegro}}
{1894}
\info{
  yeardate_ref ={\cite{Mal_90}},
  instrumentation ={ピアノ~1},
  instrumentation_ref ={\cite{Mal_90}},
}

\infotitle{2024-07-09}
{ピアノソナタ}{Piano sonata}
{1894}
\info{
  yeardate_ref ={\cite{Mal_90}},
  premier ={1894-09-xx},
  premier_ref ={\cite[44]{Mal_90}},
  instrumentation ={ピアノ~1},
  instrumentation_ref ={\cite{Mal_90}},
  supplement ={未完\supcite[45]{Mal_90}.\cite[45]{Mal_90}は冒頭7小節の譜例を掲載している.},
}

\infotitle{2024-07-09}
[Sonata_for_orchestra]{管弦楽のためのソナタ}{\textit{Sonata for orchestra}}
{1894}
\info{
  yeardate_ref ={\cite{Mal_90}},
  premier ={1894-09-xx},
  premier_ref ={\cite[44]{Mal_90}},
  instrumentation ={管弦楽(詳細不明)},
  instrumentation_ref ={\cite{Mal_90}},
  supplement ={初演はピアノ版\supcite{Mal_90}.管弦楽版は未初演?},
}

\infotitle{2024-07-09}
{弦楽四重奏\ ハ長調}{String quartet in C major}
{1894}
\info{
  yeardate_ref ={\cite{Mal_90}},
  instrumentation ={弦楽四重奏(詳細不明)},
  supplement ={断片.\supcite{Mal_90}},
}

\infotitle{2024-07-09}
{弦楽四重奏\ ニ長調}{String quartet in D minor}
{1894}
\info{
  yeardate_ref ={\cite{Mal_90}},
  instrumentation ={弦楽四重奏(詳細不明)},
  supplement ={断片.\supcite{Mal_90}},
}

\infotitle{2024-07-09}
{4つのヴァイオリンのための四重奏曲}{Quartet for 4 violins}
{1894}
\info{
  yeardate_ref ={\cite{Mal_90}},
  instrumentation ={ヴァイオリン~4},
}

\infotitle{2024-07-18}
{2つのヴァイオリンのための変奏組曲}{Suite of variations for two violins}
{1894}
\info{
  yeardate_ref ={\cite{Mal_90}},
  instrumentation ={ヴァイオリン~2},
}

\infotitle{2024-07-20}
{ピアノのためのポルカ}{Polka for piano}
{1895}
\info{
  yeardate_ref ={\cite{Mal_90}},
  instrumentation ={ピアノ~1},
}

\infotitle{2024-07-20}
{4手ピアノのためのソナチナ}{Sonatina for piano 4 hands}
{1895}
\info{
  yeardate_ref ={\cite{Mal_90}},
  instrumentation ={4手ピアノ(詳細不明)},
}

\infotitle{2024-07-20}
{ロマンス}{\textit{Romance}}
{1895}
\info{
  yeardate_ref ={\cite{Mal_90}},
  instrumentation ={4手ピアノ(詳細不明)},
}

\infotitle{2024-07-20}
{習作交響曲第1番}{Study symphony no.\@ 1}
{1895}
\info{
  yeardate_ref ={\cite{Mal_90}},
  premier ={1934-02-15},
  premier_ref ={\cite[50]{Mal_90}},
  instrumentation ={管弦楽(詳細不明)},
  supplement ={1895年3月19日の時点で少なくとも第1楽章は完成しているので\supcite{Mal_90},この周辺で作曲が完了した可能性が高い.},
  recording_url ={
    \description
      \item[第1楽章] \url{https://www.youtube.com/watch?v=1I50mRTAo44}
      \item[第2楽章] \url{https://www.youtube.com/watch?v=ZKjToUw_lB4}
      \item[第3楽章] \url{https://www.youtube.com/watch?v=n3dcCIgLSBQ}
      \item[第4楽章] \url{https://www.youtube.com/watch?v=3yOSYmlblhI}
    \enddescription
  },
  recording_info = {\cite[tracks~2--5]{Elec_98}.ルーマニア国立放送管弦楽団(管弦楽),Horia Andreescu(指揮).},
}

\infotitle{2024-07-20}
{サウルの幻影}{\textit{Vision de Saül}}
{1895}
\info{
  yeardate_ref ={\cite{Mal_90}},
  instrumentation ={管弦楽(詳細不明)},
  instrumentation_ref ={\cite{Mal_90}},
}

\infotitle{2024-07-15}
{『子ぎつね』による12の変奏曲}{12 variations on `Fuchs, du hast die Gans gestohlen'}
{1???}
\info{
  instrumentation ={ピアノ~1},
  instrumentation_ref ={\cite[42]{Mal_90}},
  supplement ={書く予定だとエネスクは記しているが,実際に書かれたかは不明で,冗談で記しただけの可能性がある.Theodor Fuchsに献呈.\supcite[42]{Mal_90}},
}

\myprintbibliography

\end{document}
